\documentclass{article}
\usepackage{listings}
\usepackage{pdfpages}
\usepackage{graphicx}
\begin{document}
\title{Modelos y Optimizaci\'on I\\ \large{Trabajo pr\'actico: Problema Combinatorio}}
\author{de Valais, Ezequiel (94463)\and	Rozanec, Matias (97404)}
\date{Octubre 2017}
\maketitle
\newpage
% FIN PRESENTACION

% Indice
\tableofcontents
\newpage

% Consigna
%\part{Consigna}
\includepdf{EnunciadoProblemaCombinatorio.pdf}
\newpage

% Resolucion

% Intro
\part{Descripci\'on de la situaci\'on problem\'atica}
Se trata de un problema de combinatoria, en el que habr\'a que incluir variables continuas y booleanas. \\
En esta instancia podemos afirmar que habr\'a que considerar una variable \textit{latencia} que deber\'a ser una variable continua; as\'i como una variable booleana que indique si un determinado datacenter se encuentra instalado en un estado espec\'ifico.

% Objetivo
\part{Objetivo}
Determinar en qu\'e estados van a estar los 3 nuevos datacenters ($DB, DC, DD$) durante un per\'iodo de tiempo para minimizar la latencia global del sistema.

% Hipotesis
\part{Hip\'otesis y supuestos}
\begin{itemize}
	\item Se instalar\'an los 3 datacenters puesto que cada datacenter agregado va a reducir la latencia.
	\item Se tomar\'a un punto en cada estado. El mismo ser\'a referente para calcular las distancias entre estados y las respectivas latencias. No hay opci\'on de instalar un datacenter en otro punto del estado que el mencionado.
	\item Para el c\'alculo se consideran \'unicamente las distancias, no la cantidad de usuarios por estado. O expresado de otra forma: para el modelo, la distribuci\'on de usuarios es uniforme a lo largo de todos los estados y en cada uno de los estados. 
	\item No puede haber dos datacenters en un mismo estado.
	\item Los costos de instalaci\'on y mantenimiento de datacenters, as\'i como cualquier otro costo que pueda implicar la instalaci\'on de los mismos, no ser\'an tenidos en cuenta por el modelo.
\end{itemize}
\newpage

% Variables
\part{Definici\'on de variables, incluyendo unidades}
\begin{itemize}
	\item $L_{i}$ : variable continua que indica la latencia (en ms) correspondiente al estado i. $\forall i \in [1, 48]$
	\item $DC_{i}$ : variable continua que indica la distancia (en millas) del datacenter $C$ al estado $i$. (\'idem para datacenters $D$ y $E$. $DA_{i}$ y  $DB_{i}$ son datos conocidos). $\forall i \in [1, 48]$
	\item $YA_{i}$ : variable bivalente que vale 0 cuando el estado $i$ tiene la menor latencia. (\'idem $B,C,D,E$)
	\item $YCe_{i}$ : variable bivalente que vale 1 cuando el datacenter $C$ est\'a en el Estado $i$. (\'idem $D, E$)
\end{itemize}
% Constantes
\section*{Constantes}
\begin{itemize}
\item $D_{ij}$: distancia entre estado $i$ y estado $j$.
\item $M$: valor mayor a cualquier distancia posible. Su valor se definir\'a al momento de pasar el modelo a software.
\end{itemize}

\part{Modelo de Programaci\'on Lineal Continua}

\begin{equation}
	\sum_{i} YA{i} + YB_{i} + YC_i + YD_i + YE_i = 1
\end{equation}

\section{Menor Distancia}
Cada $L_i$ tendr\'a como cota superior la distancia al datacenter m\'as pr\'oximo, y como cota inferior esa misma distancia en el caso de que el estado $i$ tenga la menor distancia.\\
$M \rightarrow \infty$
\begin{equation}
DA_{i} - M * YA_{i} \leq L_{i} \leq DA_{i}, \quad \forall i \in [1, 48]
\end{equation}
\begin{equation}
DB_{i} - M * YB_{i} \leq L_{i} \leq DB_{i}, \quad \forall i \in [1, 48]
\end{equation}
\begin{equation}
DC_{i} - M * YC_{i} \leq L_{i} \leq DC_{i}, \quad \forall i \in [1, 48]
\end{equation}
\begin{equation}
DD_{i} - M * YD_{i} \leq L_{i} \leq DD_{i}, \quad \forall i \in [1, 48]
\end{equation}
\begin{equation}
DE_{i} - M * YE_{i} \leq L_{i} \leq DE_{i}, \quad \forall i \in [1, 48]
\end{equation}

\section{Asociaci\'on de Datacenter a estado}
Se asegura de que cada datacenter pueda ser asignado \'unicamente a un estado.
\begin{equation}
\sum_{i=1}^{48} YCe_{i}  = 1
\end{equation}
\begin{equation}
\sum_{i=1}^{48} YDe_{i}  = 1
\end{equation}
\begin{equation}
\sum_{i=1}^{48} YEe_{i}  = 1
\end{equation}

\section{Asociaci\'on de distancia de un estado al datacenter correspondiente}
\begin{equation}
DC_{i} = 
\sum_{j=1}^{48} D_{ij} * YCe_{j}          
\end{equation}
\begin{equation}
DD_{i} = 
\sum_{j=1}^{48} D_{ij} * YDe_{j}          
\end{equation}
\begin{equation}
DE_{i} = 
\sum_{j=1}^{48} D_{ij} * YEe_{j}, \quad \forall i \in [1, 48]
\end{equation}

\part{Funcional}
\begin{equation}
Z(min) =
\sum_{i=1}^{48} L_{i} 
\end{equation}

\newpage
% HEURISTICA
\part{Heur\'istica}
\lstset { %
	language=C++,
	backgroundcolor=\color{black!5}, % set backgroundcolor
	basicstyle=\footnotesize,% basic font setting
	linewidth=14cm,
	showstringspaces=true,
	numbers=left,
	numberstyle=\tiny,
	breaklines=true,
	tabsize=4,
}
\lstinputlisting[frame=single]{Heuristica/heuristic.cpp}

\end{document}
